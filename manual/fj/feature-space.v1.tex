%feature-space.v1
As seen in \Fig{fig:pipeline}, computing the low-dimensional
feature-space representation of the data involves the following
steps:

\begin{enumerate}
  \item Generate a re-sample of the fMRI session.
  \item Compute the functional connectivity for the resample.
  \item Compute the orthogonal basis for the resample.
  \item Retain or discard basis vectors using bootstrap analysis of
  stability\cite{Bellec2010}.
  \item Project each fMRI volume (time-point) into the low
  dimensional feature space obtained above.
\end{enumerate}


The algorithms that provide this functionality are extremely
computation intensive though easily parallelized, and are therefore
implemented in MATLAB$^\circledR$ with Star-P$^\circledR$. To run
this module, it essential to have Star-P$^\circledR$ installed and
configured with MATLAB. Also, it is recommended that a cluster or
high-end high memory multi-processor computer be used. For an fMRI
session with $\approx 500$ volumes at $3$mm resolution, a minimum of
32GB RAM is essential.

A example of feature-space computation and projection operation code
fragment is:
\begin{verbatim}
    % load the preprocessed fMRI session
    vi_original = spm_vol( epi_series_preprocessed_fnames );
    % load a mask for the ROI
    vi_mask = spm_vol( roi_mask_fname);
    basis_filename = {};
    num_boot_iter = 1000;
    % BOOTSTRAP : iterate over multiple resamples
    for boot_iter = 1 : num_boot_iter
        % select a resample
        if boot_iter == 1
            % keep the first resample as the original sample
            vi_resample = vi_original;
        else
            [vi_resample] = fsResample( vi_original,
                                        ``block'', 10  );
        end
        % compute the connectivity matrix for the resample
        fsComputeConnectivity(vi_resample, vi_mask, 4,
                                0.25, connectivity_filename.bin);
        % compute the orthogonal basis vectors for the resample
        basis_filename{boot_iter} = [basis_filename_template,
                num2str(boot_iter)];
        fsOrthogonalize(connectivity_filename.bin,
                                basis_filename{boot_iter});
    end
    % perform the bootstrap analysis of stability
    [basis_set_idx, num_basis] =
                                fsBASC(basis_filename, 0.8, 0.75);
    % project original data on basis
    [basis_coords]=  fsProjectBasis( vi_original, basis_set_idx
                                        basis_filename{1});
\end{verbatim}

\section{Resampling the fMRI Session}
Function Syntax:
\begin{verbatim}
    [vi_resample] = fsResample( vi_original, resamp_type,
    option_1, option_2  );
\end{verbatim}
Arguments:
\begin{itemize}
  \item \verb"vi_original" - The array of \verb"spm_vol" structures of the original EPI time-series.
  \item \verb"resamp_type" - Valid choices are \verb"``block''" which selects the block bootstrap method
  described in \cite{Janoos2011} or   \verb"``wavelet''" that selects the wavelet based
  time-series bootstrapping method described in \cite{Patel2006}.
  \item \verb"option_1" - For \verb"resamp_type=``block''",
  \verb"option_1" is the length of a block in TR units. Typically
  \verb"option_1=10" has been found to be suitable for different
  datasets. The reader is referred to \cite{Janoos2010g} for more
  details on the tradeoffs involved. For
  \verb"resamp_type=``wavelet''", \verb"option_1" is the size of
  the coarsest scale in the wavelet decomposition tree, in TR units.
  Wavelet coefficients over a support greater than this value are
  not computed for permutation.
  \item \verb"option_2" - Not defined for
  \verb"resamp_type=``block''".
  For \verb"resamp_type=``wavelet''", \verb"option_2" is the order
  (number of vanishing moments) of the wavelet basis. A typical value is
  between 2 and 5.
\end{itemize}
Returns:
\begin{itemize}
  \item \verb"vi_resample" - The resample of the session.
\end{itemize}


\section{Computing Functional Connectivity for a Resample}
Function Syntax:
\begin{verbatim}
    fsComputeConnectivity( vi_resample, fwhm,
                        hac_frac, output_filename)
\end{verbatim}
Arguments:
\begin{itemize}
  \item \verb"fwhm" - The FWHM (in mm) of the Gaussian filter during the
  HAC step, typically set to 4.
  \item \verb"hac_frac" - The number of HAC clusters as a
  fraction of the number original voxels.
  \item \verb"output_filename" - The name of the output file
  in which to store the connectivity matrix. The matrix is
  stored row-wise in binary format (single precision float).
\end{itemize}
 Notes: This function is implemented in \verb"C++" with a
MATLAB hook.

\section{Computing the Orthogonal Basis Vectors}
Function Syntax:
\begin{verbatim}
    fsOrthogonalize(connectivity_filename_input,
                                basis_filename);
\end{verbatim}
Arguments:
\begin{itemize}
  \item \verb"connectivity_filename_input" -
    The file containing the connectivity matrix.
  \item \verb"basis_filename" - The file name (prefix) of
  the output files in which the basis vectors are stored (in zipped 4D NII
  format). Example, \verb"/data/session/subject_basis_resample_4"
  will result in the $N$ basis vectors   corresponding to the 4-th
  resample of the session being stored in the file
  \verb"/data/session/subject_basis_resample_4.nii.gz".
\end{itemize}


\section{Bootstrap Analysis of Stability}
\label{sec:basc}
 Function Syntax:
\begin{verbatim}
    [basis_set_idx, num_basis] =
                        fsBASC(basis_filename_array, tau, p);
\end{verbatim}
Arguments:
\begin{itemize}
  \item \verb"basis_filename_array" -
    A cell array of filenames of the 4D zipped NII (\verb".nii.gz")
    each containing the basis vectors computed for a resample of the
    session.
  \item \verb"tau" - The threshold at which correlations between the
  corresponding basis vectors across resamples are retained.
  Typically set between 0.6 and 0.9.
  \item \verb"p" - The probability value of suprathreshold correlations,
  if when exceeded the corresponding basis vector is retained in the
  final low-dimensional set.
\end{itemize}
Returns:
\begin{itemize}
  \item \verb"basis_set_idx" -
   The indices of the basis vector retained.
  \item \verb"num_basis" - The number of basis vectors retained
\end{itemize}

\section{Project Session on to Basis}
Function Syntax:
\begin{verbatim}
    [fs_coords]=  fsProjectBasis( vi_original,
                basis_set_idx, basis_filename);
\end{verbatim}
Arguments:
\begin{itemize}
    \item \verb"vi_original" - The array of \verb"spm_vol" structures of the original EPI time-series.
 \item \verb"basis_set_idx" -
   The indices of the basis vector retained.
  \item \verb"basis_filename" -
    The filename of the 4D zipped NII (\verb".nii.gz")
    containing the basis vectors computed for a particular re-sample of the
    session.
\end{itemize}
Returns:
\begin{itemize}
  \item \verb"fs_coords" - An number of scans times number of basis
  array giving the coordinates of each fMRI time-frame in the low
  dimensional feature space.
\end{itemize}
