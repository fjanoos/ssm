% @desc Packages.tex
% @note Packages and new latex commands defined and used throughout the text
% @author firdaus janoos
% @date 2:52 PM Tuesday, October 20, 2009
%%%%%%%%%%%%%%%%%%%%%%%%%%%%%%%%%%%%%%%%%%%%%%%%%%%%%%%%%%%%%%%%%%%%%%%%%%%%%%%


\usepackage{verbatim}
\usepackage{graphicx}
\usepackage{epstopdf}
\usepackage{ifthen}
\usepackage{makeidx}  % allows for indexgeneration
\usepackage{url}

\usepackage[usenames]{color}

\usepackage{amsmath}   % From the American Mathematical Society
\usepackage{amsfonts}% to get the \mathbb alphabet
\usepackage{amssymb}


%\usepackage{mathptmx} % don't use thsi crap package !!

%This package provides some new list environments. Itemized and enumerated lists
%can be typeset within paragraphs, as paragraphs and in a compact version.  Most
%environments have optional arguments to format the labels.  Additionally, the L
%A TEX environments itemize and enumerate can be extended to use a similar
%optional argument.
%The option pointedenum and the macro \pointedenum format the labels as in �1.�,
%�1.1.�,  �1.1.1.�   and  �1.1.1.1.�   and  the  references  without  the
%trailing point.  The option pointlessenum and the macro \pointlessenum do not
%use the trailing point in the labels either.
%The environment asparaenum is an enumerated environment in which the asparaenum
%items are formatted as separate paragraphs.

%\usepackage[pointedenum]{paralist}
%\setdefaultenum{i}{i.i}{i.i.i}{i.i.i.i}

%\setdefaultenum{i.}{i.}{i.}{i.}
%\newenvironment{flushenum}{
%\begin{enumerate}
%  \setlength{\leftmargin}{0pt}
%}
%{\end{enumerate}
%}

\usepackage{enumerate}
\renewcommand{\labelenumi}{\roman{enumi}}
\renewcommand{\labelenumii}{\roman{enumi}.\roman{enumii}}
\renewcommand{\labelenumiii}{\roman{enumi}.\roman{enumii}.\roman{enumiii}}
\renewcommand{\labelenumiv}{\roman{enumi}.\roman{enumii}.\roman{enumiii}.\roman{enumiv}}



\usepackage{verbatim}
%The verbatim environment, \begin{verbatim} ... \end{verbatim},
%permits us to insert large sections of preformated text in a LaTeX
%file. It is very handy for inserting large chunks of code in a
%document.
%If the verbatim package is not quite what you need check the
%moreverb package.It provides a large number of verbatim-like
%environments.

\usepackage{textcomp}
\usepackage{dsfont}     %\mathds fonts
\usepackage{psfrag}     % This package allows you to substitute LaTeX
                        % commands for text in imported EPS graphic files.
\usepackage{subfigure}  % This package makes it easy to put subfigures
                        % in your figures. i.e. "figure 1a and 1b"
\usepackage{url}        % Provides better support for handling and breaking
                        % URLs. url.sty is already installed on most LaTeX
\usepackage{stfloats}   % Gives LaTeX2e the ability to do double column
                        % floats at the bottom of the page as well as the top.
\usepackage{times}
\usepackage{ifpdf}
\usepackage{listings}

%\usepackage{underscore}

\usepackage{setspace}
%% Double Spacing: Double spacing is made with the setspace package. Once the
%setspace package is loaded, you just have to specify the type of spacing you
%want at the start of your document. The command appears as such for double
%spacing: % \doublespacing % Instead of doublespacing, this same command and
%syntax can be used to create single spacing with the \singlespacing or to make
%one half spacing using the command \onehalfspacing % You can switch from
%doublespacing to singlespacing by simply changing the command to \singlespacing
%before the spacing you want switched to singlespacing (this is very helpful
%when switching from the main text to the notes and indices). % If a different
%spacing is required, then the \setstretch{baselinestretch} command can be used
%in the preamble to set the baselinestretch appropriately. The default spacing
%with this style option is single spacing. The syntax for this command is
%\setstretch{baselinestretch}


%Smarter version of the original latex2e cross reference
%commands. Generated strings are customizable, Babel options are recognized
%(further info in the package documentation). \vref, \vpageref, \vref is similar
%to \ref but adds an additional page reference, like on the facing page or on
%page 27 whenever the corresponding \label is not on the same page. \vpageref is
%a variation of \pageref with similar functionality. \vrefrange, \vpagerefrange,
%The \v...range commands take two labels as arguments and produce strings which
%depend on wether or not these labels fall onto a single page or on different
%pages. \vref*, \vpageref*, \vpagerefrange*, Star * variants do not add any
%space before the generated text for situations like:
\usepackage{varioref}
%\renewcommand{\ref}{\vref} %if using varioref



% Referring to labels in other documents
\usepackage{xr}
\usepackage{xr-hyper}
%Setting     \externaldocument{volume1} will load all the references from volume1 into your present document.
%To have the facilities of xr working with hyperref, you need xr-hyper. For
%simple hyper-cross-referencing (i.e., to a local PDF file you�ve just
%compiled), write:
%
%    \usepackage{hyperref}
%    \externaldocument[V1-]{volume1}
%    ...
%    ... the \nameref{V1-introduction})...
%
%and the name reference will appear as an active link to the �introduction�
%chapter of volume1.pdf.
% SEE: http://www.tex.ac.uk/cgi-bin/texfaq2html?label=extref

%HyperTeX is a defacto standard for inclusion of hyperlink information in TeX
%and LaTeX documents, which can then be used to generated PDF or DVI files with
%embedded links. These links enable easy navigation inside documents as well as
%between documents, just as in HTML, using standard PDF, Postscript and DVI
%viewers. Various TeX, LaTeX and BibTeX style sheets and macro packages exist
%which facilitate the construction of hyperlinked documents.
\usepackage[plainpages=false,pdfpagelabels,pagebackref=false]{hyperref}

%\usepackage{cite}
\usepackage[numbers,square,sort]{natbib} %for author-date citations
%\renewcommand{~\cite}{~\citep} %if using natbib
%% Natbib options can be
%% provided with \biboptions{...} command. Following options are
%% valid:

%%   round  -  round parentheses are used (default)
%%   square -  square brackets are used   [option]
%%   curly  -  curly braces are used      {option}
%%   angle  -  angle brackets are used    <option>
%%   semicolon  -  multiple citations separated by semi-colon
%%   colon  - same as semicolon, an earlier confusion
%%   comma  -  separated by comma
%%   numbers-  selects numerical citations
%%   super  -  numerical citations as superscripts
%%   sort   -  sorts multiple citations according to order in ref. list
%%   sort&compress   -  like sort, but also compresses numerical citations
%%   compress - compresses without sorting
%%
%% \biboptions{comma,round}
%\biboptiona{numbers,square,sort}

%A separate bibliography for each �chapter� of a document can be provided with
%the package chapterbib (which comes with a bunch of other good bibliographic
%things). The package allows you a different bibliography for each \included
%file (i.e., despite the package�s name, the availability of bibliographies is
%related to the component source files of the document rather than to the
%chapters that logically structure the document).
%\usepackage[sectionbib]{chapterbib}

\usepackage{booktabs}
%The package uses \toprule for the uppermost rule (or line), \midrule  for the rules
%appearing in the middle of the table (such as under the header), and
%\bottomrule for the lowermost rule. This ensures that the rule weight and
%spacing are acceptable. In addition, \cmidrule  can be used for mid-rules that
%span specified columns
%http://tug.ctan.org/macros/latex/contrib/booktabs/booktabs.pdf

\usepackage{tabularx}
%This package provides a table environment called tabularx which is similar to
%the tabular* environment, except that it has a new column specifier X  (in
%uppercase). The column(s) specified with this specifier will be stretched to
%make the table as wide as specified, greatly simplifying the creation of
%tables.
% makes the columns center justified
%\renewcommand{\tabularxcolumn}[1]{>{\footnotesize}m{#1}}




%However, two extension packages provide support  for  typesetting  tables
%across  multiple  pages, namely longtable [3] and supertabular [1].
\usepackage{longtable}

%\usepackage{fullpage} % improves the page space layout of the article class
\usepackage{layout}
\usepackage{fancybox}
\usepackage{fancyhdr}
\usepackage{floatflt}
\usepackage{parskip}
\usepackage{wrapfig}
%The LaTeX wrapfig package
%The wrapfig package allows text to be wrapped around floating objects at the
%side of the page. The package provides two environments, wrapfigure and
%wraptable. These environments are not regular floats and may print out of
%sequence, but accompanying captions are correctly numbered. The package has one
%option available through the Options and Packages command on the Typeset menu
%that controls whether to print information in the .log file.

%% The lineno packages adds line numbers. Start line numbering with
%% \begin{linenumbers}, end it with \end{linenumbers}. Or switch it on
%% for the whole article with \linenumbers after \end{frontmatter}.
\usepackage{lineno}

\usepackage{rotating}


%Package dirtree allows to display directory tree, like in the windows ex-
%plorator.
\usepackage{dirtree}

\newcommand{\fjnote}[1]{{\color{BrickRed}{#1}}}

% see http://en.wikibooks.org/wiki/LaTeX/Algorithms_and_Pseudocode
\usepackage{algorithmic}
%\usepackage{algorithm}
%\numberwithin{algorithm}{section}  % <--- chapter, section etc. depending on what is required
\usepackage[lined,boxed,vlined,linesnumbered,algochapter]{algorithm2e}
% add algorithms to the TOC

\usepackage{amsthm} %theorem environment (http://en.wikibooks.org/wiki/LaTeX/Theorems)
\newtheorem{theorem}{Theorem}
\newtheorem{lemma}{Lemma}
\newtheorem{definition}{Definition}
\numberwithin{equation}{section}

%% The numcompress package shorten the last page in references.
%% `nodots' option removes dots from firstnames in references.
%% `nocompress' option prevent shortening of last page as
%% by default it will shorten.
\usepackage[nodots,nocompress]{numcompress}

%% SIDE BAR
\newenvironment{mysidebar}[1]
{%begin  part  of  env.
\begin{figure}[t!]
\small
\rule{\linewidth}{2pt}
%\vspace{-16pt}
\centerline{\textbf{#1}}
\rule{\linewidth}{2pt}
\\
}
{%end  part  of  env.
\\
\rule{\linewidth}{1pt}
%\rule[.13in]{\linewidth}{2.5pt}
\end{figure}
}

\newcommand{\fnsz}[1]{{\footnotesize #1}}
\newcommand{\smsz}[1]{{\small #1}}


%% to number subsubsections
\setcounter{secnumdepth}{3}

\newcommand{\cf}{\S}
\newcommand{\eqn}[1]{eqn.~\ref{#1}}
\newcommand{\Sec}[1]{Section~\ref{#1}}
\newcommand{\Chap}[1]{Chapter~\ref{#1}}
\newcommand{\Fig}[1]{Fig.~\ref{#1}}
\newcommand{\App}[1]{Appendix~\ref{#1}}
\newcommand{\Algo}[1]{Algorithm~\ref{#1}}

\newcommand{\B}{\mathbb{B}}
\newcommand{\C}{\mathbb{C}}
\newcommand{\D}{\mathrm{D}}
\newcommand{\f}{\mathbf{f}}
\newcommand{\F}{\mathrm{F}}
\newcommand{\h}{{\mathbf{h}}}
\newcommand{\I}{\mathrm{I}}
\newcommand{\HH}{\mathrm{H}}
\newcommand{\LL}{\mathcal{L}}
\newcommand{\M}{\mathcal{M}}
\newcommand{\N}{\mathcal{N}}
\newcommand{\Oo}{\mathcal{O}}
\newcommand{\Q}{\mathcal{Q}}
\newcommand{\rr}{\mathbf{r}}
\newcommand{\R}{\mathbb{R}}
\newcommand{\s}{{\mathbf{s}}}
\newcommand{\Ss}{\delta{{\mathbf{Z}}}}
\newcommand{\Ssmall}{\delta{{\mathbf{z}}}}
\newcommand{\uu}{\mathbf{u}}
\newcommand{\vv}{\mathbf{v}}
\newcommand{\V}{\mathrm{V}_\epsilon}
\newcommand{\VV}{\sigma}
\newcommand{\W}{\mathbf{W}}
\newcommand{\w}{\mathbf{w}}
\newcommand{\wbar}{\mathbf{\omega}}
\newcommand{\x}{{x}}
\newcommand{\X}{{x}}
\newcommand{\Xn}{\X^{(n)}}
\newcommand{\y}{{\mathbf{Y}}}
\newcommand{\Y}{{\mathbf{y}}}
\newcommand{\z}{{\mathbf{Z}}}
\newcommand{\Z}{{\mathbf{z}}}

%transpose operation
\newcommand{\tr}{^\top}
\newcommand{\TD}{\mathrm{FD}}
\newcommand{\Tr}{\mathrm{Tr}}
\newcommand{\diag}{\mathrm{diag}}
\newcommand{\half}{\frac{1}{2}}

\newcommand{\E}{\mathds{E}}
\newcommand{\Var}{\mathds{V}\text{ar}}
\newcommand{\Cov}{\mathds{C}\text{ov}}
\newcommand{\Corr}{\mathds{C}\text{orr}}

\newcommand{\Prr}{\mathrm{Pr}}
\newcommand{\p}{p_\theta}
\newcommand{\phat}{p_\widehat{\theta}}
\newcommand{\pn}{p_{\theta^{(n)}}}
\newcommand{\pp}[1]{p_{\theta^{(#1)}}}
\newcommand{\qn}{q^{(n)}}
\newcommand{\qq}{q^*}

\newcommand{\LogPs}{\mathrm{LogPs}}
\newcommand{\LogDiff}{\mathrm{LogDiff}}
\newcommand{\Ph}{\mathrm{Ph}}
\newcommand{\len}{\mathrm{Len}}
\newcommand{\Ans}{\mathrm{Ans}}
